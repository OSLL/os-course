\begin{frame}
\frametitle{Проблемы блокирующей синхронизации}
\begin{itemize}
  \item С использованием взаимного исключения может быть связан ряд проблем:
  \begin{itemize}
    \item конвоирование и инверсия приоритетов;
    \item отсутвие "композицируемости".
    \item deadlock-и;
  \end{itemize}
\end{itemize}
\end{frame}

\begin{frame}
\frametitle{Композиция модулей}
\begin{itemize}
  \item Один из основных инженерных принципов состоит в том, чтобы собирать
  сложные системы из простых компонентов:
  \begin{itemize}
    \item вы разбиваете свой код на фукнции/классы;
    \item функции и классы группируются в библиотеки;
    \item из библиотек и соединяющего их кода создается приложение.
  \end{itemize}
  \item Принцип работает до тех пор пока компоненты сравнительно "просты" и
  "независимы":
  \begin{itemize}
    \item каждый из них решает конкретную задачу, и решение можно проверить;
    \item собирая приложение из корректных компонет мы получаем корректное
    приложение (с кучей но...).
  \end{itemize}
\end{itemize}
\end{frame}

\begin{frame}[fragile]
\frametitle{Композиция локов}
\begin{columns}
  \begin{column}{.5\linewidth}
    \begin{lstlisting}
size_t squeue_size(struct squeue *q)
{
    size_t size;

    lock(&q->lock);
    size = queue_size(&q->queue);
    unlock(&q->lock);
    return size;
}

struct node *squeue_pop(struct squeue *q)
{
    struct node *node;

    lock(&q->lock);
    node = queue_pop(&q->queue);
    unlock(&q->lock);
    return node;
}

int main()
{
    struct squeue queue;
    ...
    if (squeue_size(&queue))
        process(squeue_pop(&queue));
    ...
}
    \end{lstlisting}
  \end{column}
  \begin{column}{.6\linewidth}
    \begin{itemize}
      \item Локи не компонуются:
      \begin{itemize}
        \item т. е. нельзя ожидать, что пару корректных, с точки зрения
        многопоточности, функций можно объединить в новую корректную функцию;
        \item ответсвенность за использование локов ложится на пользователя.
      \end{itemize}
    \end{itemize}
  \end{column}
\end{columns}
\end{frame}

\begin{frame}
\frametitle{Deadlock}
\begin{itemize}
  \item Deadlock - состояние потока, в котором он не может прогрессировать
  \begin{itemize}
    \item пример взаимной блокировки потоков: \emph{Thread 0} держит блокировку
    \emph{Lock 0}, а поток \emph{Thread 1} держит блокировку \emph{Lock 1};
    \item представим, что \emph{Thread 0} попытается взять блокировку
    \emph{Lock 1}, и одновременно \emph{Thread 1} попытается взять
    \emph{Lock 0};
    \item в deadlock-е может участвовать более 2 потоков и 2 блокировок.
  \end{itemize}
  \item Deadlock-и может быть трудно обнаружить:
  \begin{itemize}
    \item как и многие другие проблемы связанные с конкурентностью, они могут
    проявляться очень редко;
    \item для воспроизведения может потребоваться учесть очень много факторов.
  \end{itemize}
\end{itemize}
\end{frame}

\begin{frame}
\frametitle{Средства борьбы с deadlock-ами}
\framesubtitle{Страусинный алгоритм}
\begin{itemize}
  \item Самый простой способ называется "страусинный алгоритм":
  \begin{itemize}
    \item если ПО не критично к надежности;
    \item если deadlock проявляется редко;
    \item то можно просто забить!
  \end{itemize}
  \item Классический пример cost/benefit анализа:
  \begin{itemize}
    \item поиск причин проблемы потребует много ресурсов (чем реже он
    проявляется тем тяжелее найти причины);
    \item выгода от решения проблемы минимально - раз в неделю мользователь
    может перезапустить компьютер/программу;
    \item явно не наш метод!
  \end{itemize}
\end{itemize}
\end{frame}

\begin{frame}
\frametitle{Средства борьбы с deadlock-ами}
\framesubtitle{Ограниченное ожидание}
\begin{itemize}
  \item Можно ограничить ожидание при вызове lock:
  \begin{itemize}
    \item если блокировку не удалось захватить в течении, скажем, минуты, то
    можно подозревать deadlock;
    \item можно собрать максимум возможной информации (backtrace-ы, информация
    о блокировке, информация о потоке, который ее держит) и вывести сообщение.
  \end{itemize}
  \item Отличный инженерный вариант:
  \begin{itemize}
    \item его не трудно реализовать;
    \item отсутсвие deadlock-ов не гарантируется, но если он возникнет у вас
    будет максимум полезной информации.
  \end{itemize}
\end{itemize}
\end{frame}

\begin{frame}
\frametitle{Средства борьбы с deadlock-ами}
\framesubtitle{Достоверное определение deadlock-а}
\begin{itemize}
  \item Долгое ожидание не гарантирует deadlock, но deadlock можно определить
  достоверно:
  \begin{itemize}
    \item построим ориентированный граф, в котором каждый узел соответствует
    потоку;
    \item ребрам графа будут соответствовать незавершенные вызовы lock;
    \item наличие циклов в таком графе означает deadlock;
    \item сам цикл (потоки и блокировки) - ценная отладочная информация.
  \end{itemize}
  \item Имея такой граф зависимостей мы можем "исправлять" deadlock-и:
  \begin{itemize}
    \item для этого достаточно "удалить" одно из ребер в цикле;
    \item можно сообщить потоку, что захват блокировки не удался;
    \item это поможет только, если поток может адекватно обработать такую
    ситуацию.
  \end{itemize}
\end{itemize}
\end{frame}

\begin{frame}
\frametitle{Средства борьбы с deadlock-ами}
\framesubtitle{Граф исторических зависимостей}
\begin{itemize}
  \item Как находить deadlock-и до того как они произойдут?
  \begin{itemize}
    \item давайте определять зависимости между блокировками в виде
    ориентированного графа;
    \item вершинами графа будут различные блокировки;
    \item из блокировки \emph{A} в блокировку \emph{B} в графе  ведет ребро,
    если был поток, который пытался захватить \emph{A} с захваченной \emph{B};
    \item мы можем запустить наше приложение и построить такой граф.
  \end{itemize}
  \item Если в графе исторических зависимостей есть цикл, то, \emph{вероятно},
  у вас в программе есть deadlock
  \begin{itemize}
    \item при этом не учитывается количество потоков, и какой поток какие
    зависимости в графе породил.
  \end{itemize}
\end{itemize}
\end{frame}

\begin{frame}
\frametitle{Средства борьбы с deadlock-ами}
\framesubtitle{Формальная верификация/Model Checking}
\begin{itemize}
  \item Формальная верификация позволяет строго формально проверить вашу
  программу на соответствие определенным требованиям:
  \begin{itemize}
    \item на самом деле не программу, а модель вашей программы;
    \item модель программы может быть построена автоматически;
    \item построенные автоматически модели, как правило, слишком большие и на
    них метод не работает;
    \item требования должны задаваться математической формулой, как правило
    формулой в одной из темпоральных логик.
  \end{itemize}
  \item Формальная верификация работает в критически важных системах
  \begin{itemize}
    \item \href{https://ti.arc.nasa.gov/m/tech/rse/publications/papers/ASE00/jpf2-ase.pdf}{Model Checking Programs}
  \end{itemize}
\end{itemize}
\end{frame}
