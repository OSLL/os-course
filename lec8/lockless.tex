\begin{frame}
\frametitle{Синхронизация без блокировок}
\begin{itemize}
  \item Возможно ли синхронизовывать потоки без блокировки?
  \begin{itemize}
    \item в простых случаях да, очевидно, мы можем атомарно обновлять некоторые,
    переменные используя RMW операции, без всяких блокировок;
    \item если алгоритмы будут обходиться без взаимного исключения, то и проблем
    связанных с блокировками не будет;
    \item впрочем, возможно, будут другие.
  \end{itemize}
  \item Что значит без блокировок? Есть три вида гарантий прогресса:
  \begin{itemize}
    \item obstruction-free;
    \item lock-free;
    \item wait-free.
  \end{itemize}
\end{itemize}
\end{frame}

\begin{frame}
\frametitle{Obstruction freedom}
\begin{itemize}
  \item Алгоритм/структура данных/etc может называться obstruction free если
  выполняется следующее условие:
  \begin{itemize}
    \item если мы останавливаем \emph{любые} потоки кроме одного в
    \emph{произвольном} месте работы этих потоков;
    \item то оставшийся поток гарантированно достигнет прогресса (завершит
    операцию над разделяемыми данными).
  \end{itemize}
  \item Если потоки используют взаимное исключение, то они не obstruction-free:
  \begin{itemize}
    \item если остановить поток когда он держит блокировку, то другие потоки не
    смогут прогрессировать.
  \end{itemize}
\end{itemize}
\end{frame}

\begin{frame}
\frametitle{Lock freedom}
\begin{itemize}
  \item Алгоритм/структура данных/etc может называться lock free если
  выполняется следующее условие:
  \begin{itemize}
    \item \emph{один из потоков} гарантированно прогрессирует, независимо от
    состояния дургих;
    \item т. е. даже если мы начнем останавливать случайные потоки в случайных
    местах их исполнения, один из оставшихся обязательно достигнет прогресса.
  \end{itemize}
  \item Lock freedom сильнее obstruction freedom:
  \begin{itemize}
    \item obstruction freedom гарантирует прогресс только если дать одному из
    потоков исполняться мнонпольно достаточно долго;
    \item lock freedom гарантирует, что какой-то из потоков точно прогрессирует,    возможно, за счет неудачи других потоков.
  \end{itemize}
\end{itemize}
\end{frame}

\begin{frame}
\frametitle{Wait freedom}
\begin{itemize}
  \item Алгоритм/структура данных/etc может называться wait free если
  выполняется следующее условие:
  \begin{itemize}
    \item каждый поток гарантированно достигает прогресса за ограниченное
    количество шагов, независимо от других потоков.
  \end{itemize}
  \item Wait freedom сильнее lock freedom:
  \begin{itemize}
    \item lock freedom гарантирует прогресс как минимум одного потока, а wait
    freedom гарантирует прогресс всех потоков.
  \end{itemize}
\end{itemize}
\end{frame}
