\begin{frame}
\frametitle{Разделяемая память}
\begin{itemize}
  \item Несколько потоков могут обрабатывать общие данные
  \begin{itemize}
    \item не редко несколько потоков могут решить общую задачу быстрее.
  \end{itemize}
  \item Неатомарные чтения/обновления общих данных могут приводить к проблемам:
  \begin{itemize}
    \item можно получить неконсистентное состояние разделяемой памяти;
    \item не предсказуемый результат работы.
  \end{itemize}
\end{itemize}
\end{frame}

\begin{frame}
\frametitle{Атомарность}
\begin{itemize}
  \item Операция атомарна, если она может только выполниться полностью или не
  выполниться вовсе:
  \begin{itemize}
    \item либо все эффекты операции видны, либо не видны никакие;
    \item наблюдатель не может увидеть только часть результата операции;
    \item наблюдатель не может заметить, что операция была приостановлена.
  \end{itemize}
  \item Не все что выглядит как одна операция является атомарным!
\end{itemize}
\end{frame}

\begin{frame}[fragile]
\frametitle{Пример}
\begin{columns}
  \begin{column}{.45\linewidth}
    \begin{lstlisting}
int main() {
    static volatile int i = 42;

    ++i;
    return 0;
}
    \end{lstlisting}
  \end{column}
  \begin{column}{.45\linewidth}
    \begin{lstlisting}
main:
  mov 0x200c4a(%rip),%eax
  add $0x1,%eax
  mov %eax,0x200c41(%rip)
  xor %eax,%eax
  retq
    \end{lstlisting}
  \end{column}
\end{columns}
\begin{itemize}
  \item Обычная операция инкремента транслируется в три ассемблерные инструкции:
  \begin{itemize}
    \item чтение, обновление (инкремент) и запись;
    \item что если инкремент \emph{одной переменной} пытаются сделать сразу
    несколько потоков?
  \end{itemize}
\end{itemize}
\end{frame}

\begin{frame}[fragile]
\frametitle{Состояние гонки по данным}
\begin{columns}
  \begin{column}{.3\linewidth}
    \begin{lstlisting}
i: .int 42

thread0:
  mov i, %eax
  add $1, %eax
  mov %eax, i
  retq

thread1:
  mov i, %eax
  add $1, %eax
  mov %eax, i
  retq
    \end{lstlisting}
  \end{column}
  \begin{column}{.7\linewidth}
    \begin{itemize}
      \item thread0 и thread1 вызываются из разных потоков;
      \item каждую инструкцию считаем атомарной;
      \item какое значение будет в \emph{i} после завершения?
    \end{itemize}
  \end{column}
\end{columns}
\end{frame}

\begin{frame}[fragile]
\frametitle{Хороший вариант исполнения}
\begin{columns}
  \begin{column}{.45\linewidth}
    \begin{lstlisting}[basicstyle=\small]
thread0:
  mov i, %eax
  add $1, %eax
  mov %eax, i



  retq
_
    \end{lstlisting}
  \end{column}
  \begin{column}{.45\linewidth}
    \begin{lstlisting}[basicstyle=\small]
thread1:



  mov i, %eax
  add $1, %eax
  mov %eax, i

  retq 
    \end{lstlisting}
  \end{column}
\end{columns}
\end{frame}

\begin{frame}[fragile]
\frametitle{Плохой вариант исполнения}
\begin{columns}
  \begin{column}{.45\linewidth}
    \begin{lstlisting}[basicstyle=\small]
thread0:
  mov i, %eax

  add $1, %eax
  mov %eax, i


  retq
_
    \end{lstlisting}
  \end{column}
  \begin{column}{.45\linewidth}
    \begin{lstlisting}[basicstyle=\small]
thread1:

  mov i, %eax


  add $1, %eax
  mov %eax, i

  retq 
    \end{lstlisting}
  \end{column}
\end{columns}
\end{frame}

\begin{frame}
\frametitle{Состояние гонки}
\begin{itemize}
  \item Ситуация, когда результат работы зависит от порядка в котором
  выполняются атомарные инструкции называется состоянием гонки
  \begin{itemize}
    \item состояние гонки может приводить к случайным ошибкам;
    \item порядок, в котором атомарные инструкции разных потоков будут выполнены
    зависит от целой кучи обстоятельств:
    \begin{itemize}
      \item от входных данных;
      \item от скорости CPU;
      \item от количества работы на CPU (от прерываний, других потоков и др.).
    \end{itemize}
  \end{itemize}
  \item \emph{Пока} атомарными инструкциями для нас будут только чтение и запись
  переменных
  \begin{itemize}
    \item размер переменной при этом ограничен.
  \end{itemize}
\end{itemize}
\end{frame}
