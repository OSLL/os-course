\begin{frame}
\frametitle{Критическая секция}
\begin{itemize}
  \item Участок кода, в котором происходит обращение к общим данным и для
  которого важен порядок наложения, будем называть критической секцией
  \begin{itemize}
    \item чтобы избежать состояния гонки достаточно чтобы в критической секции
    не могло находиться более одного потока одновременно;
    \item другими словами, если критическая секция атомарна, то проблем нет.
  \end{itemize}
\end{itemize}
\end{frame}

\begin{frame}
\frametitle{Взаимное исключение 1/4}
\begin{itemize}
  \item Взаимное исключение (Mutual Exclusion, aka mutex) - примитив, который
  позволяет оградить критическую секцию и сделать ее атомарной
  \begin{itemize}
    \item пара функций lock и unlock
    \begin{itemize}
      \item вызываем lock перед тем как войти в критическую секцию;
      \item вызываем unlock после выхода из критической секции.
    \end{itemize}
  \end{itemize}
\end{itemize}
\end{frame}

\begin{frame}
\frametitle{Взаимное исключение 2/4}
\begin{itemize}
  \item Не более одного потока может находиться между lock и unlock
  единовременно (и, соответственно, в критической секции)
  \begin{itemize}
    \item если в критической секции находится другой поток, то lock не вернет
    управление, до тех пор пока поток в критической секции не вызовет unlock;
  \end{itemize}
  \item если в критической секции не находится ни один поток, то из нескольких
  конкурирующих вызовов lock из разных потоков как минимум один окажется
  успешным
  \begin{itemize}
    \item в противном случае мы могли бы просто сделать бесконечный цикл в
    lock - это избавляет от гонки, но создает несколько других проблем.
  \end{itemize}
\end{itemize}
\end{frame}

\begin{frame}
\frametitle{Взаимное исключение 3/4}
\begin{itemize}
  \item Корректность гарантируется только для потоков использующих mutex
  \begin{itemize}
    \item т. е. все потоки должны использовать lock/unlock, чтобы все было
    хорошо;
    \item т. е. приходится полагаться, на то, что потоки не делают гадостей.
  \end{itemize}
  \item Кроме того можем считать, что ни один поток не "сидит" в критической
  секции бесконечно долго
  \begin{itemize}
    \item т. е. если поток вызвал lock, то он точно когда-нибудь вызовет unlock;
    \item не сильное ограничение, учитывая, что мы и так считаем потоки
    хорошими;
    \item приходится следить, чтобы поток не "упал" в критической секции.
  \end{itemize}
\end{itemize}
\end{frame}

\begin{frame}
\frametitle{Взаимное исключение 4/4}
\begin{itemize}
  \item Мы не можем делать предположений
  \begin{itemize}
    \item о скорости работы потоков
    \begin{itemize}
      \item мы знаем, что время нахождения в критической секции конечно, но
      конкретное ограничение сверху нам не известно и мы не можем на него
      полагаться;
    \end{itemize}
    \item о взаимной скорости работы потоков
    \begin{itemize}
      \item мы не можем считать, что один поток быстрее другого, или что их
      скорости работы;
    \end{itemize}
    \item другими словами, мы не можем делать предположений о том, как
    инструкции из разных потоков "накладываются" друг на друга.
  \end{itemize}
\end{itemize}
\end{frame}

\begin{frame}[fragile]
\frametitle{Реализация взаимного исключения}
\framesubtitle{Глобальный флаг 1/2}
\begin{columns}
  \begin{column}{.45\linewidth}
    \begin{lstlisting}
extern int claim1;
int claim0;

void lock0(void)
{
  claim0 = true;
  while (claim1);
}

void unlock0(void)
{
  claim0 = false
}
    \end{lstlisting}
  \end{column}
  \begin{column}{.45\linewidth}
    \begin{lstlisting}
extern int claim0;
int claim1;

void lock1(void)
{
  claim1 = true;
  while (claim0);
}

void unlock1(void)
{
  claim1 = false;
}
    \end{lstlisting}
  \end{column}
\end{columns}
\end{frame}

\begin{frame}[fragile]
\frametitle{Реализация взаимного исключения}
\framesubtitle{Глобальный флаг 2/2}
\begin{columns}
  \begin{column}{.45\linewidth}
    \begin{lstlisting}[basicstyle=\small]
void lock0(void)
{
  claim0 = true;

  while (claim1);
}
    \end{lstlisting}
  \end{column}
  \begin{column}{.45\linewidth}
    \begin{lstlisting}[basicstyle=\small]
void lock1(void)
{

  claim1 = true;
  while (claim0);
}
    \end{lstlisting}
  \end{column}
\end{columns}
\begin{itemize}
  \item Оба потока могут установить свои флаги до того, как другой поток
  успеет их проверить
  \begin{itemize}
    \item оба потока будут ждать в цикле бесконечно - такая ситуация называется
    deadlock-ом.
  \end{itemize}
\end{itemize}
\end{frame}

\begin{frame}[fragile]
\frametitle{Реализация взаимного исключения}
\framesubtitle{Глобальный порядок 1/2}
\begin{columns}
  \begin{column}{.45\linewidth}
    \begin{lstlisting}
int turn;

void lock0(void)
{
  while (turn != 0);
}

void unlock0(void)
{
  turn = 1;
}
    \end{lstlisting}
  \end{column}
  \begin{column}{.45\linewidth}
    \begin{lstlisting}
extern int turn;

void lock1(void)
{
  while (turn != 1);
}

void unlock1(void)
{
  turn = 0;
}
    \end{lstlisting}
  \end{column}
\end{columns}
\end{frame}

\begin{frame}[fragile]
\frametitle{Реализация взаимного исключения}
\framesubtitle{Глобальный порядок 2/2}
\begin{columns}
  \begin{column}{.45\linewidth}
    \begin{lstlisting}
int turn;

void lock0(void)
{
  while (turn != 0);
}

void unlock0(void)
{
  turn = 1;
}
    \end{lstlisting}
  \end{column}
  \begin{column}{.45\linewidth}
    \begin{lstlisting}
extern int turn;

void lock1(void)
{
  while (turn != 1);
}

void unlock1(void)
{
  turn = 0;
}
    \end{lstlisting}
  \end{column}
\end{columns}
\begin{itemize}
  \item Что если lock0 никогда не будет вызван?
  \begin{itemize}
    \item тогда и unlock0 никогда не будет вызван;
    \item turn никогда не будет равен 1;
    \item lock1 никогда не вернет управление.
  \end{itemize}
\end{itemize}
\end{frame}

\begin{frame}[fragile]
\frametitle{Реализация взаимного исключения}
\framesubtitle{Глоабльный порядок и глоабльный флаг 1/2}
\begin{columns}
  \begin{column}{.45\linewidth}
    \begin{lstlisting}
int claim0;
extern int claim1;
int turn;

void lock0(void)
{
  claim0 = true;
  turn = 1;

  while (claim1 && turn != 0);
}

void unlock0(void)
{
  claim0 = flase;
}
    \end{lstlisting}
  \end{column}
  \begin{column}{.45\linewidth}
    \begin{lstlisting}
extern int claim0;
int claim1;
extern int turn;

void lock1(void)
{
  claim1 = true;
  turn = 0;

  while (claim0 && turn != 1);
}

void unlock1(void)
{
  claim1 = flase;
}
    \end{lstlisting}
  \end{column}
\end{columns}
\end{frame}

\begin{frame}[fragile]
\frametitle{Реализация взаимного исключения}
\framesubtitle{Глоабльный порядок и глоабльный флаг 2/3}
\begin{columns}
  \begin{column}{.45\linewidth}
    \begin{lstlisting}
void lock0(void)
{
  claim0 = true;
  turn = 1;

  while (claim1 && turn != 0);
}
    \end{lstlisting}
  \end{column}
  \begin{column}{.45\linewidth}
    \begin{lstlisting}
void lock1(void)
{
  claim1 = true;
  turn = 0;

  while (claim0 && turn != 1);
}
    \end{lstlisting}
  \end{column}
\end{columns}
\begin{itemize}
  \item Если в lock0 мы прочитали claim1 и он оказался равен false, то
  \begin{itemize}
    \item другой поток не вызывал lock1 вообще и не присваивал $turn = 0$,
    т. е. поток вызвавший lock0 - единственный кандидат;
    \item другой поток не сможет войти до тех пор пока мы не вызовем unlock0,
    т. к. claim0 равен true и другой поток сам присвоит $turn = 0$ в lock1.
  \end{itemize}
\end{itemize}
\end{frame}

\begin{frame}[fragile]
\frametitle{Реализация взаимного исключения}
\framesubtitle{Глоабльный порядок и глоабльный флаг 3/3}
\begin{columns}
  \begin{column}{.45\linewidth}
    \begin{lstlisting}
void lock0(void)
{
  claim0 = true;
  turn = 1;

  while (claim1 && turn != 0);
}
    \end{lstlisting}
  \end{column}
  \begin{column}{.45\linewidth}
    \begin{lstlisting}
void lock1(void)
{
  claim1 = true;
  turn = 0;

  while (claim0 && turn != 1);
}
    \end{lstlisting}
  \end{column}
\end{columns}
\begin{itemize}
  \item Если в lock0 мы прочитали claim1 и он оказался равен true, то
  \begin{itemize}
    \item lock1 либо еще не присвоил в turn значение 0
    \begin{itemize}
      \item значит он скоро сделает это, и мы будем ждать пока lock1 не выполнит
      $turn = 0$, после чего сможем выйти из цикла;
    \end{itemize}
    \item либо сделал это до того, как lock0 присвоил $turn = 1$
    \begin{itemize}
      \item значит claim0 и claim1 оба равны true и turn равен 1, значит lock1
      рано или поздно выйдет из цикла, в то время как lock0 будет ждать в цикле.
    \end{itemize}
  \end{itemize}
\end{itemize}
\end{frame}

\begin{frame}[fragile]
\frametitle{Алгоритм Петерсона}
\begin{columns}
  \begin{column}{.45\linewidth}
    \begin{lstlisting}
void lock0(void)
{
  turn = 1;
  claim0 = true;

  while (claim1 && turn != 0);
}
    \end{lstlisting}
  \end{column}
  \begin{column}{.45\linewidth}
    \begin{lstlisting}
void lock1(void)
{
  turn = 0;
  claim1 = true;

  while (claim0 && turn != 1);
}
    \end{lstlisting}
  \end{column}
\end{columns}
\begin{itemize}
  \item В алгоритме очень важен порядок присваиваний в claim0/claim1 и turn:
  \begin{itemize}
    \item допустим lock0 присвоил $turn = 1$, затем в дело вступил lock1;
    \item lock1 присвоил $turn = 0$ и затем $claim1 = true$, после чего
    прочитал значение claim0 и увидел там false - цикл в lock1 завершился;
    \item lock0 продолжил там где остановился, т. е. присвоил $claim0 = true$,
    далее lock0 проверяет условие цикла и видит turn равным 0, т. к. lock1
    записал туда последним, и тоже выходит из цикла.
  \end{itemize}
\end{itemize}
\end{frame}

\begin{frame}[fragile]
\frametitle{Алгоритм Петерсона для N потоков 1/4}
\begin{lstlisting}
int claim[N];
int turn[N - 1];

void lock(int thread_id)
{
  for (int level = 0; level < N - 1; ++level) {
    claim[thread_id] = level + 1;
    turn[level] = thread_id;

    while (1) {
      int found = false;
      for (int thread = 0; !found && thread != N; ++thread) {
        if (thread == thread_id) continue;
        found = claim[thread] > level;
      }
      if (!found) break;
      if (turn[level] != thread_id) break;
    }
  }
}

void unlock(int thread_id)
{
  claim[thread_id] = 0;
}
\end{lstlisting}
\end{frame}

\begin{frame}
\frametitle{Алгоритм Петерсона для N потоков 2/4}
\begin{itemize}
  \item у нас есть $N - 1$ глобaлных флага в массиве turn;
  \item и $N$ переменных намерения в массиве claim - по одной для каждого
  потока;
  \item изначально все перемнные claim равны 0.
\end{itemize}
\end{frame}

\begin{frame}
\frametitle{Алгоритм Петерсона для N потоков 3/4}
\begin{itemize}
  \item Выбор потока входящего в критическую секцию то же самое, что и выбор
  потоков, которые будут ждать:
  \begin{itemize}
    \item если у нас три потока, то нужно отсеять два и оставить один;
    \item отсеивание будет происходить в два этапа: на нулевом этапе отсеивается
    один из потоков, на первом второй;
    \item переменная claim для потока хранит номер этапа (+ 1), до которого он
    добрался;
    \item каждая переменная turn соответствует своему этапу и хранит
    идентификатор последнего потока добравшегося до этого этапа.
  \end{itemize}
\end{itemize}
\end{frame}

\begin{frame}
\frametitle{Алгоритм Петерсона для N потоков 4/4}
\begin{itemize}
  \item Поток \emph{может} перейти к следующему этапу если выполняется хотя бы
  одно из условий:
  \begin{itemize}
    \item нет потоков на том же этапе или впереди него, т. е. поток уже
    победил - цикл проверяющий значения claim всех потоков отвечает за это
    условие;
    \item поток не последний поток на данном этапе, т. е. пришел новый поток
    и записал в соответствующий turn свой идентификатор - проверка turn в код
    отвечает за это условие;
  \end{itemize}
\end{itemize}
\end{frame}

\begin{frame}
\frametitle{Порядок входа в критическую секцию}
\begin{itemize}
  \item Если сразу несколько потоков конкурируют за lock, то в каком порядке
  они их получат?
  \begin{itemize}
    \item нам бы хотелось, чтобы каждый из кандидатов завершил lock рано или
    поздно;
    \item но не хотелось бы, чтобы ожидание завершения lock было не
    ограниченным;
    \item еще больше не хотелось бы ждать, пока другие входят и выходят из
    критической секцию.
  \end{itemize}
\end{itemize}
\end{frame}

\begin{frame}
\frametitle{Честность алгоритма Петерсона, пример}
\tiny\begin{enumerate}
  \item Изначально все равно 0:
  \\$claim[0] == claim[1] == claim[2] == 0$ и $turn[0] == turn[1] == 0$;
  \item поток 0 вызывает lock:
  \\$claim[0] == 1$, $claim[1] == claim[2] == 0$ и $turn0 == turn1 == 0$;
  \item поток 1 вызывает lock:
  \\$claim[0] == claim[1] == 1$, $claim[2] == 0$, $turn[0] == 1$ и
  $turn[1] == 0$;
  \item поток 2 вызывает lock:
  \\$claim[0] == claim[1] == claim[2] == 1$, $turn[0] == 2$ и $turn[1] == 0$;
  \item \alert{поток 1 не последний вошел на этап 0, а значит он может
  двинуться дальше:
  \\$claim[0] == claim[2] == 1$, $claim[1] == 2$, $turn[0] == 2$ и
  $turn[1] == 1$;}
  \item впереди потока 1 или на том же этапе других потоков нет - поток 1
  выиграл, он входит в критическую секцию, а затем выходит из нее:
  \\$claim[0] == claim[2] == 1$, $claim[1] == 0$, $turn[0] == 2$ и
  $turn[1] == 1$;
  \item поток 1 опять вызывает lock:
  \\$claim[0] == claim[1] == claim[2] == 1$, $turn[0] == 1$ и $turn[1] == 1$;
  \item поток 2 не последний на этапе 0 - он может перейти к следующему этапу:
  \\$claim[0] == claim[1] == 1$, $claim[2] == 2$, $turn[0] == 1$ и
  $turn[1] == 2$;
  \item впереди потока 2 или на том же этапе никоого нет - поток 2 выиграл, он
  входит в критическую секцию, а затем выходит из нее:
  \\$claim[0] == claim[1] == 1$, $claim[2] == 0$, $turn[0] == 1$ и
  $turn[1] == 2$;
  \item поток 2 опять вызывает lock:
  \\$claim[0] == claim[1] == claim[2] == 1$, $turn[0] == 2$ и $turn[1] == 2$;
  \item \alert{поток 1 не последний вошел на этап 0, а значит он может
  двинуться дальше:
  \\$claim[0] == claim[2] == 1$, $claim[1] == 2$, $turn[0] == 2$ и
  $turn[1] == 1$.}
\end{enumerate}
\end{frame}

\begin{frame}
\frametitle{Честность взаимного исключения 1/2}
\begin{itemize}
  \item Разделим lock на две части:
  \begin{enumerate}
    \item вход (будем обозначать его как $D$) - эта часть всегда завершается за
    известное заранее конечное количество шагов;
    \item ожидание (будем обозначать как $W$) - может потребовать неограниченное
    количество шагов:
    \begin{itemize}
      \item зависит от того, как долго другой поток сидит в критической секции;
      \item зависит от того, в каком порядке накладываются инструкции из разных
      потоков;
    \end{itemize}
    \item после ожидания идет критическая секция (будем обозначть ее как $CS$).
  \end{enumerate}
\end{itemize}
\end{frame}

\begin{frame}
\frametitle{Честность взаимного исключения 2/2}
\begin{itemize}
  \item Свойство $r$-ограниченного ожидания:
  \begin{itemize}
    \item для некоторый потоков $A$ и $B$, из того что $D^j_A$ предшествует
    $D^k_B$ ($j$-ый вход потока $A$ предшествует $k$-ому входу потока $B$),
    следует, что $CS^j_A$ предшествует $CS^{k + r}_B$;
    \item другими словами, после того как мы завершили свой $D$, то после этого
    не более чем $r$ потоков войдут в критическую секцию (\emph{ВНИМАНИЕ: это
    очень-очень-очень грубое упрощение!}).
  \end{itemize}
  \item Для алгоритма Петерсона для N потоков не существует такого разделения
  на $D$ и $W$ чтобы он удовлетворял свойству $r$-ограниченного ожидания для
  любого $r$
  \begin{itemize}
    \item в частном случае двух потоков алгоритм Петерсона удовлетворяет
    свойству $1$-ограниченного ожидания.
  \end{itemize}
\end{itemize}
\end{frame}

\begin{frame}
\frametitle{Взаимное исключение с очередью}
\begin{itemize}
  \item Пусть все потоки, которые хотят войти в критическую секцию
  выстравиваются в очередь:
  \begin{itemize}
    \item будем выдавать каждому пришедшему наименьший из номеров больший
    всех номеров в очереди;
    \item обслуживать будем в порядке возрастания номеров.
  \end{itemize}
  \item Как найти и занять нужный номер?
  \begin{itemize}
    \item мы должны посмотреть на номера в очереди и выбрать себе подходящий;
    \item но это действие звучит очень не атомарно;
    \item будем вместо номера использовать пару: номер и идентификатор потока;
    \item даже если выбранный номер не уникален, пара будет уникальной.
  \end{itemize}
\end{itemize}
\end{frame}

\begin{frame}[fragile]
\frametitle{Алгоритм пекарни}
\begin{lstlisting}
int select[N];
int ticket[N];

int max(void)
{
  int rc = 0;
  for (int i = 0; i != N; ++i)
    if (ticket[i] > rc)
      rc = ticket[i];
  return rc;
}

void lock(int thread_id)
{
  select[thread_id] = true;
  ticket[thread_id] = max() + 1;
  select[thread_id] = false;

  for (int thread = 0; thread != N; ++thread) {
    if (thread == thread_id)
      continue;
    while (select[thread]);
    while (ticket[thread] && ((ticket[thread] < ticket[thread_id]) ||
      (ticket[thread] == ticket[thread_id] && thread < thread_id)));
  }
}

void unlock(int thread_id)
{
  ticket[thread_id] = 0;
}
\end{lstlisting}
\end{frame}

\begin{frame}
\frametitle{Честность алгоритма пекарни}
\begin{itemize}
  \item вход алгоритма пекарни $D$ состоит из выбора номера для потока;
  \item если $D$ потока $A$ предшествует $D$ потока $B$, то возможны два
  случая:
  \begin{itemize}
    \item номер выбранный потоком $B$ меньше или равен номеру выбранному потоком
    $A$, но это значит, что поток $B$ не видел номер выбранный $A$, т. е. $A$
    его уже занулил, т. е. завершил критическую секцию;
    \item номер выбранный потоком $B$ больше номера выбранного потоком $A$, а
    значит в критическую секцию он войдет позже.
  \end{itemize}
  \item Другими словами алгоритм пекрани является $0$-ограниченным
  \begin{itemize}
    \item обратите внимание, что если $D_A$ не предшествует строго $D_B$ то их
    взаимный порядок может быть любым.
  \end{itemize}
\end{itemize}
\end{frame}
