\begin{frame}
\frametitle{Нижние границы}
\begin{itemize}
  \item Для реализации взаимного исключения для N потоков с использованием
  перемнных, которые можно атомарно только читать или писать (RW регистры),
  требуется не менее N таких RW регистров
  \begin{itemize}
    \item дополнительно каждый поток может иметь внутреннее состояние, но так
    как оно не видимо другим потокам, то и принимать решения на его основе они
    не могут;
    \item размер RW регистров \emph{абсолютно не важен}, т. е. они вообще могут
    не иметь ограничения на хранимые данные.
  \end{itemize}
  \item Покажем это на примере двух потоков
  \begin{itemize}
    \item доказательство в общем виде тоже существует, но оно нам не очень
    интересно.
  \end{itemize}
\end{itemize}
\end{frame}

\begin{frame}
\frametitle{Вспомогательные утверждения 1/2}
\begin{itemize}
  \item Состояние нашей системы описывается состоянием RW регистров и внутренним
  состоянием каждого потока участника
  \begin{itemize}
    \item мы почти не можем делать предположений о внутреннем состоянии потока;
    \item но мы знаем, что если поток не делал никаких действий, то его
    внутренее состояние не могло измениться.
  \end{itemize}
\end{itemize}
\end{frame}

\begin{frame}
\frametitle{Вспомогательные утверждения 2/2}
\begin{itemize}
  \item Пусть $С$ некоторое состояние, в котором некоторый поток $p$ не
  находится в критической секции и не пытается туда войти (не вызвал еще lock);
  рассмотрим последовательность шагов $a$ такую что:
  \begin{itemize}
    \item в $a$ только $p$ что-то делает (планировщик отдал CPU потоку $p$);
    \item в конце $a$ поток $p$ находится в критической секции.
  \end{itemize}
  \item Тогда мы можем утверждать, что среди шагов $a$ есть как минимум одна
  запись потока $p$ в RW регистр.
\end{itemize}
\end{frame}

\begin{frame}
\frametitle{Доказательство 1/3}
\begin{itemize}
  \item Допустим противное, есть два потока $p_0$ и $p_1$, один общий RW регистр
  $x$, вся система находится в некотором начальном состояние $C_0$ и мы можем
  реализовать взаимное исключение.
  \item Рассмотрим последовательность шагов, в которой участвует только $p_0$,
  то из состояния $C_0$ система должна рано или поздно перейти в состояние $C$,
  в котором $p_0$ находится в критической секции
  \begin{itemize}
    \item а значит где-то между $C_0$ и $C$ было состояние, из которого мы ушли
    сделав запись в RW регистр $x$, обозначим состояние как $C'$,
    последовательность шагов переводяшую систему из $C_0$ в $C'$ обозначим как
    $a'$;
    \item последовательность шагов переводящую $C'$ в $C$ обозначим как $a$.
  \end{itemize}
\end{itemize}
\end{frame}

\begin{frame}
\frametitle{Доказательство 2/3}
\begin{itemize}
  \item Если же мы дадим выполняться только $p_1$, то рано или поздно система
  должна прийти в состояние $C''$, в котором $p_1$ находится в критической
  секции
  \begin{itemize}
    \item последовательность шагов переводящуюю систему из $C_0$ в $C''$
    обозначим $a''$.
  \end{itemize}
\end{itemize}
\end{frame}

\begin{frame}
\frametitle{Доказательство 3/3}
\begin{itemize}
  \item Мы можем "склеить" последовательности шагов $a'$, $a''$ и $a$:
  \begin{itemize}
    \item последовательность шагов $a'$ не пишет в $x$, т. е. с точки зрения
    потока $p_1$ состояния $C_0$ и $C'$ не различимы;
    \item значит от $C'$ мы можем отложить последовательность шагов $a''$ и
    перейти в состояние $C''$, в котором $p_1$ находится в критической секции;
    \item внутреннее состояние потока $p_0$ в $C'$ и $C''$ идентично, так как
    он не делал шагов в $a''$, а значит от $C''$ мы модем отложить $a$;
    \item первое действие в $a$ - запись в $x$, которая затирает все видимые
    результаты работы потока $p_1$, значит поток $p_0$ продолжит так как будто
    $p_1$ ничего не делал и войдет в критическую секцию в результате $a$.
  \end{itemize}
\end{itemize}
\end{frame}
