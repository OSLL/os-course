\begin{frame}[fragile]
\frametitle{Алгоритм Петерсона: то, что написали вы}
\begin{lstlisting}
int turn;
int claim[2];

void lock(int thread)
{
	const int other = (thread + 1) & 1;

	clain[thread] = 1;
	turn = other;

	while (claim[other] && turn != thread);
}

void unlock(int thread)
{
	claim[thread] = 0;
}
\end{lstlisting}
\end{frame}

\begin{frame}[fragile]
\frametitle{Алгоритм Петерсона: то, что уидел компилятор}
\begin{lstlisting}
.comm claim, 8

lock:
	/* eax = thread + 1 */
	leal	1(%rdi), %eax
	/* rdx = thread */
	movslq	%edi, %rdx
	/* claim[thread] = 1 */
	movl	$1, claim(,%rdx,4)
	/* eax = eax & 1, aka, eax = other */
	andl	$1, %eax

	/* rdx = other */
	movslq	%eax, %rdx

	/* check if claim[other] == 0 */
	movl	claim(,%rdx,4), %edx
	testl	%edx, %edx
	je	exit

	/* check if other == thread */
	cmpl	%eax, %edi
	je	exit

loop:	jmp	loop

exit:	ret
\end{lstlisting}
\end{frame}

\begin{frame}
\frametitle{Почему???}
\begin{itemize}
  \item Почему компилятор выкинул turn?
  \begin{itemize}
    \item Компилятор не увидел смысла в сохранении turn и удалил его.
  \end{itemize}
  \item Почему компилятор сравнивает other с thread?
  \begin{itemize}
    \item Компилятор выкинул turn, other - это значение, которое должно было
    лежать в turn.
  \end{itemize}
  \item Почему компилятор проверяет условия только один раз?
  \begin{itemize}
    \item Компиялтор не видит, что turn или claim могут измениться, а значит
    достаточно проверить их один раз и либо выйти либо зациклиться.
  \end{itemize}
\end{itemize}
\end{frame}

\begin{frame}
\frametitle{Компилятор виноват?}
\begin{itemize}
  \item Компилятор работает в терминах наблюдаемого поведения:
  \begin{itemize}
    \item компилятор может делать с вашей программой все что угодно, до тех пор
    пока наблюдаемое поведение сохраняется;
    \item компилятор хочет делать с вашей программой много чего, потому что он
    должен оптимизировать код.
  \end{itemize}
  \item Как сделать операции над данными частью наблюдаемого поведения?
  \begin{itemize}
    \item в языках C и C++ для этого есть ключевое слово \emph{volatile};
    \item компилятор воздерживается от многих оптимизаций кода работающего над
    volatile объектами.
  \end{itemize}
\end{itemize}
\end{frame}

\begin{frame}[fragile]
\frametitle{Оптимизации компилятора и volatile}
\begin{itemize}
  \item Пометив данные как volatile вы связываете компилятору руки везде, где
  вы используете volatile данные
  \begin{itemize}
    \item причем везде - и там где это важно и там где нет;
    \item есть ли более точечные средства ограничения фантазии компилятора?
    \item обычно они есть, но они зависят от компилятора.
  \end{itemize}
  \item Барьеры компиялтора:
  \begin{itemize}
    \item например, в gcc можно использовать такую конструкцию
\begin{lstlisting}
__asm__ volatile ("" : : : "memory")
\end{lstlisting}
  \end{itemize}
\end{itemize}
\end{frame}

\begin{frame}[fragile]
\frametitle{Алгоритм Петерсона с барьерами компиялтора}
\begin{lstlisting}
int turn;
int claim[2];

void lock(int thread)
{
	const int other = (thread + 1) & 1;

	claim[thread] = 1;
	__asm__ volatile ("" : : : "memory"); // order is important here
	turn = other;

	do {
		// reread claim and turn after every iteration
		__asm__ volatile ("" : : : "memory");
	} while (claim[other] && turn != thread);
}

void unlock(int thread)
{
	claim[thread] = 0;
}
\end{lstlisting}
\end{frame}

\begin{frame}
\frametitle{Замечания про volatile}
\begin{itemize}
  \item volatile не может являться средством синхронизации:
  \begin{itemize}
    \item volatile не дает никаких гарантий атомарности, т. е. работа с volatile
    переменными все еще не атомарна;
    \item стандарты C и C++ вводят такое понятие как \emph{точка следования},
    между двумя точками следования операциям компиялтору разрешено
    переупорядочивать обращения к volatile объектам;
    \item стандарты не определяют строго, что такое "обращение" к volatile
    объекту;
    \item компиялтор - не единственный источник переупорядочивания/оптимизаций,
    т. е. только компиялторных средств не достаточно.
  \end{itemize}
\end{itemize}
\end{frame}
