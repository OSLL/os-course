\begin{frame}[fragile]
\frametitle{Store Buffer 1/2}
\begin{columns}
  \begin{column}{0.3\linewidth}
    \begin{lstlisting}
#define barrier() \
  __asm__ volatile( "" \
	: \
	: \
	: "memory")

void foo(void)
{
  a = 1;
  barrier();
  b = 1;
}

void bar(void)
{
  while (b == 0) {
    barrier();
    continue;
  }
  barrier();
  assert(a == 1);
}
    \end{lstlisting}
  \end{column}
  \begin{column}{0.8\linewidth}
  \begin{itemize}
    \item Посмотрим на функцию foo:
    \begin{itemize}
      \item допустим b лежит в кеше CPU, а a нет;
      \item CPU пытается присвоить a, но ее нет в кеше, он посылает Invalidate и
      должен дождаться ответа;
      \item но CPU ведь уже знает, что будет лежать в a, ему даже не нужно
      старое значение, так почему вдруг он должен ждать?
    \end{itemize}
  \end{itemize}
  \end{column}
\end{columns}
\end{frame}

\begin{frame}
\frametitle{Store Buffer 2/2}
\begin{itemize}
  \item Если мы не можем писать прямо в кеш, пока не придет Invalidate Ack, то
  заведем специальный некогерентный буфер для таких записей:
  \begin{itemize}
    \item такой буфер обычно называют Store Buffer;
    \item как только мы получаем Invalidate Ack мы переносим данные из Store
    Buffer в кеш;
    \item сохранив данные в Store Buffer CPU может продолжать работать.
  \end{itemize}
  \item Не ломает ли Store Buffer что-нибудь?
  \begin{itemize}
    \item а если сообщения могут переупорядочиваться?
  \end{itemize}
\end{itemize}
\end{frame}

\begin{frame}[fragile]
\frametitle{Теперь CPU виноват? 1/2}
\begin{columns}
  \begin{column}{0.3\linewidth}
    \begin{lstlisting}
#define barrier() \
  __asm__ volatile( "" \
	: \
	: \
	: "memory")

void foo(void)
{
  a = 1;
  barrier();
  b = 1;
}

void bar(void)
{
  while (b == 0) {
    barrier();
    continue;
  }
  barrier();
  assert(a == 1);
}
    \end{lstlisting}
  \end{column}
  \begin{column}{0.8\linewidth}
  \begin{itemize}
    \item Рассмотрим следующий порядок событий:
    \begin{itemize}
      \item CPU0 выполняет foo, а CPU1 выполняет bar;
      \item переменная a хранится только в кеше CPU1, а b только в кеше CPU0;
      \item CPU0 исполняет $a = 1$, переменной нет в кеше, записываем в Store
      Buffer и посылаем Invalidate;
      \item CPU1 проверяет условие цикла, переменной b нет в кеше, посылаем
      Read;
      \item CPU0 выполняет $b = 1$, b уже в кеше, можно прям там и обновить;
      \item CPU0 получает Read и отсылает $b == 1$ в ответ;
      \item CPU1 получает значение $b == 1$, и выполняет assert.
    \end{itemize}
  \end{itemize}
  \end{column}
\end{columns}
\end{frame}

\begin{frame}[fragile]
\frametitle{Теперь CPU виноват? 2/2}
\begin{columns}
  \begin{column}{0.3\linewidth}
    \begin{lstlisting}
#define barrier() \
  __asm__ volatile( "" \
	: \
	: \
	: "memory")

void foo(void)
{
  a = 1;
  barrier();
  b = 1;
}

void bar(void)
{
  while (b == 0) {
    barrier();
    continue;
  }
  barrier();
  assert(a == 1);
}
    \end{lstlisting}
  \end{column}
  \begin{column}{0.8\linewidth}
  \begin{itemize}
    \item Посмотрим на функцию bar:
    \begin{itemize}
      \item допустим переменная \emph{a} находится в кеше CPU, а переменная
      \emph{b} нет;
      \item т. е. чтение \emph{a} занимает много меньше времени, чем \emph{b};
      \item если CPU будет ждать пока завершится чтение \emph{b} ему придется
      остановиться;
      \item но у него уже есть \emph{a} и он \emph{не знает} о зависимости между
      ними, так может обработаем \emph{a} вперед?
    \end{itemize}
  \end{itemize}
  \end{column}
\end{columns}
\end{frame}

\begin{frame}
\frametitle{Барьеры памяти}
\begin{itemize}
  \item Откуда вдруг взялось так много проблем?
  \begin{itemize}
    \item при многопточности, важными становятся зависимости между переменными;
    \item например, для функции bar важно, чтобы foo сначала записала a, и
    только потом b;
    \item ни компилятор ни CPU об этих зависимостях ничего не знают.
  \end{itemize}
  \item Чтобы указать CPU, какие действия нельзя "переставлять" есть специальные
  инструкции
  \begin{itemize}
    \item такие инструкции называют барьерами памяти, что конкретно они
    разершают или запрещают зависит от архитектуры.
  \end{itemize}
\end{itemize}
\end{frame}

\begin{frame}
\frametitle{Memory Order}
\begin{itemize}
  \item Правила, по которым CPU может переставлять инструкции зависят от
  архитектуры, что делать простым смертным?
  \begin{itemize}
    \item языки программирования предоставляют несколько хорошо известных
    сравнительно (ну да, конечно...) простых моделей: JMM, sequential
    consistency, acquire/consume/release;
    \item вы можете просто пользоваться обычными mutex-ами реализованными
    кем-то другим и не иметь проблем с memory order, т. к. lock и unlock
    делают такими, чтобы они использовали все нужные барьеры памяти/компилятора.
  \end{itemize}
\end{itemize}
\end{frame}
