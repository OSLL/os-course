\begin{frame}
\frametitle{Модели в распределенных системах}
\begin{itemize}
  \item Модель описывает свойства и гарантии распределенной системы и условия,
  в которых она должна функционировать
  \begin{itemize}
    \item свойства окружения;
    \item типы возможных ошибок;
    \item гарантии консистентности.
  \end{itemize}
\end{itemize}
\end{frame}

\begin{frame}
\frametitle{Процессы}
\begin{itemize}
  \item В данном контексте, процессы - это узлы распределенной системы, решающие
  совместными услилиями общую задачу
  \begin{itemize}
    \item агенты/ноды/компьютеры/etc;
    \item например, можем рассматривать разные модели ошибок для процессов:
    \begin{itemize}
      \item мы можем считать их надежными;
      \item они могут падать и, возможно, восстанавливаться после падений;
      \item византийские ошибки - совершенно произвольные ошибки/злонамеренное
      поведение.
    \end{itemize}
  \end{itemize}
\end{itemize}
\end{frame}

\begin{frame}
\frametitle{Взаимодействие процессов}
\begin{itemize}
  \item Один из самых очевидных способов взаимодействия - обмен сообщениями
  \begin{itemize}
    \item зачастую это просто обмен данными по сети.
  \end{itemize}
  \item Про обмен сообщениями полезно знать:
  \begin{itemize}
    \item могут ли они переупорядочиваться;
    \item могут ли они теряться.
  \end{itemize}
\end{itemize}
\end{frame}

\begin{frame}
\frametitle{Предположения о времени}
\begin{itemize}
  \item Есть, более или менее, две крайние модели, описывающие временные
  характеристики системы:
  \begin{itemize}
    \item синхронная модель - время доставки сообщений ограничено и граница
    известна, время на обработку сообщения ограничено и известно;
    \item асинхронная модель - время доставки и обработки сообщений не
    ограничено.
  \end{itemize}
\end{itemize}
\end{frame}
