\section{Основное задание}

В этом домашнем задании вам нужно реализовать файловую систему в памяти (что
довольно просто, но не забывайте про синхронизацию) и заполнить ее начальными
данными (initramfs, о которой далее сказано подробней).

\begin{enumerate}
  \item Реализовать файловую систему в памяти, которая поддерживает следующий
        минимальный набор операций:
        \begin{itemize}
          \item open/close - создание/открытие и закрытие файла;
          \item read/write - чтение и запись файла по указанному смещению;
          \item mkdir - создание каталога;
          \item readdir - перечисление записей каталога;
        \end{itemize}
        Конкретный набор функций реализующих эти операции остается на ваше
        усмотрение. Хорошим вариантом будет мимикрировать Unix API для работы
        с файловой системой.
  \item Распарсить образ initramfs в формате cpio в памяти и заполнить его
        содержимым файловую систему;
\end{enumerate}

Для простоты считайте, что все имена файлов и каталогов даются в абсолютном
формате, т. е. такого понятия как текущий каталог у нас нет.
