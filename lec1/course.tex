\begin{frame}
\frametitle{Структура курса}
\begin{itemize}
  \item Лекции
  \begin{itemize}
    \item посещаемость не проверяю;
    \item контрольных не устраиваю;
    \item возможность задавать вопросы - не задаете вопросы, значит все понятно.
  \end{itemize}
  \item Практики и домашние задания
  \begin{itemize}
    \item 5 домашних заданий - пишем простую ОС;
    \item домашние задания не независимы - пропустили одну будут проблемы со
    следующим;
    \item практики отведены для обсуждения домашних заданий (поэтому последняя
    пара).
  \end{itemize}
\end{itemize}
\end{frame}

\begin{frame}
\frametitle{Темы лекций 1/3}
\begin{itemize}
  \item Прерывания и внешние устройства
  \begin{itemize}
    \item тема привязана к первому домашнему заданию;
  \end{itemize}
  \item Управление памятью
  \begin{itemize}
    \item алгоритмы аллокации памяти (Buddy, SLOB/SLAB);
    \item защита памяти, сегментация, страничная организация и page fault;
    \item интерфейс ОС для процессов (как работает malloc?);
    \item NUMA?
  \end{itemize}
  \item Многозадачность, планирование и конкуретность:
  \begin{itemize}
    \item понятие процесса и потока исполнения;
    \item диспетчеризация и планирование потоков;
    \item многоядерность, когерентность кешей и барьеры памяти;
    \item примитивы синхронизации потоков;
  \end{itemize}
\end{itemize}
\end{frame}

\begin{frame}
\frametitle{Темы лекций 2/3}
\begin{itemize}
  \item Межпроцессное взаимодействие
  \begin{itemize}
    \item существующие примитивы IPC, без рассмотрения реализации;
  \end{itemize}
  \item Файловые системы
  \begin{itemize}
    \item иерархические ФС: файлы, каталоги, символьные и "жесткие" ссылки;
    \item классические файловые системы: FAT и ext2;
    \item индексные структуры данных: B-деревья и модификации, LSM;
    \item транзакции;
    \item устройство диска, избыточность и помехоустойчивое кодирование?
  \end{itemize}
\end{itemize}
\end{frame}

\begin{frame}
\frametitle{Темы лекций 3/3}
\begin{itemize}
  \item Краткое введениме в распределенные системы:
  \begin{itemize}
    \item консенсус, разрешимые и неразрешимые задачи, протоколы консенсуса.
  \end{itemize}
  \item Краткое введение в виртуализацию:
  \begin{itemize}
    \item история вопроса, назначение;
    \item критерии эффективной виртуализации;
    \item QEMU и динамическая трансляция;
    \item аппаратная поддержка виртуализации.
  \end{itemize}
\end{itemize}
\end{frame}

\begin{frame}
\frametitle{Темы домашних заданий}
\begin{itemize}
  \item прерывания (3 основных + 2 дополнительных)
  \item аллокация (3 + 2)
  \item потоки (3 + 2)
  \item ФС (2 + 0)
  \item системные вызовы (3 + 0)
\end{itemize}
\end{frame}

\begin{frame}
\frametitle{Условия выполнения заданий}
\begin{itemize}
  \item Правила выполнения домашних заданий:
  \begin{itemize}
    \item задания сдаются в виде ссылки на git репозиторий;
    \item для каждого задания есть жесткий дедлайн (опоздали на 1 минуту или на
    1 час не важно);
    \item чтобы задание было засчитано оно должно компилироваться и работать
    \emph{у меня};
    \item по вашему решению будут задаваться вопросы.
  \end{itemize}
  \item Критерии оценки:
  \begin{itemize}
    \item "отлично" - 11 баллов;
    \item "хорошо" - 8 баллов;
    \item "удовлетворительно" - 5 баллов.
  \end{itemize}
\end{itemize}
\end{frame}

\begin{frame}
\frametitle{Необходимые утилиты}
\begin{itemize}
  \item QEMU (виртуальная машина, далее просто ВМ)
  \begin{itemize}
    \item тестировать ОС быстрее в ВМ;
    \item к QEMU можно подключить отладчик;
  \end{itemize}
  \item GNU ld, gcc или clang
  \begin{itemize}
    \item владельцы Windows могут использовать ВМ c Linux;
    \item владельцы Mac OS имееют две возможности:
    \begin{itemize}
      \item собрать toolchain из исходников;
      \item использовать ВМ с Linux;
    \end{itemize}
  \end{itemize}
  \item задание №0 - настроить окружение, собрать пример и запустить;
\end{itemize}
\end{frame}
