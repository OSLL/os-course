\begin{frame}
\frametitle{Разделяемая память}
\begin{itemize}
  \item Каждый процесс имеет собственное адресное пространство по-умолчанию
  \begin{itemize}
    \item в этом и заключается основа безопасности процессов.
  \end{itemize}
  \item Однако по требованию процессы могут создавать участки общей памяти:
  \begin{itemize}
    \item обмен данными через общую память позволяет избежать копирования при
    передаче данных;
    \item однако не предоставляет по-умолчанию возможности синхронизации (вам
    нужно реализовать синхронизацию самостоятельно).
  \end{itemize}
\end{itemize}
\end{frame}

\begin{frame}
\frametitle{Разделяемая память}
\begin{itemize}
  \item Чтобы попросить ОС создать/удалить именованный участок памяти
  используются shm\_open/shm\_unlink (open/unlink - создание/удаление памяти)
  \begin{itemize}
    \item участок памяти хранится до удаления или перезагрузки.
  \end{itemize}
  \item Чтобы иметь доступ к разделяемой памяти как к обычной памяти необходимо
  отобразить ее в адресное пространство процесса
  \begin{itemize}
    \item для отображения региона используется mmap;
    \item munmap - обратная к mmap операция (удаляет отображение);
    \item участок разделяемой памяти в разных процессах может быть отображен в
    разные места адресного пространства.
  \end{itemize}
\end{itemize}
\end{frame}

\begin{frame}[fragile]
\frametitle{Пример}
\lstinputlisting[firstline=14,lastline=38,language=C]{src/shmem/shmem.c}
\end{frame}
