\begin{frame}
\frametitle{Сигналы}
\begin{itemize}
  \item Сигналы - это сообщения, которые получает процесс в некоторых
  предвиденных и не очень случаях:
  \begin{itemize}
    \item кто-то послал процессу сигнал с помощью вызова kill (говорящее
    название);
    \item процесс сам себе послал сигнал с помощью вызова abort (опять же
    говорящее название);
    \item процесс сделал какую-то гадость:
    \begin{itemize}
      \item доступ к не своей/не существующей/не выровненной памяти;
      \item попытка выполнить недопустимую инструкцию;
    \end{itemize}
    \item другие менее/более противные ситуации.
  \end{itemize}
\end{itemize}
\end{frame}

\begin{frame}
\frametitle{Сигналы}
\begin{itemize}
  \item При получении сигнала процесс (или скорее от имени процесса) выполняется
  некоторое действие:
  \begin{itemize}
    \item типичные действия - проигнорировать сигнал, упасть или упасть красиво
    с core dump-ом;
    \item для некоторых типов сигналов действия можно переопределить: например
    мы можем перехватить CTRL+C.
  \end{itemize}
\end{itemize}
\end{frame}

\begin{frame}[fragile]
\frametitle{Пример}
\lstinputlisting[firstline=18,lastline=40,language=C]{src/sigaction/sigaction.c}
\end{frame}

\begin{frame}[fragile]
\frametitle{Пример}
\lstinputlisting[firstline=7,lastline=16,language=C]{src/sigaction/sigaction.c}
\end{frame}

\begin{frame}
\frametitle{Ограничения на обработку сигналов}
\begin{itemize}
  \item Обработчики сигналов вызываются асинхронно (в некотором смысле)
  относительно кода потока:
  \begin{itemize}
    \item не все функции можно безопасно вызывать из обработчика сигнала;
    \item например, printf нельзя, по крайней мере так говорит POSIX.
  \end{itemize}
  \item POSIX требует чтобы довольно много функций были signal-safe
  \begin{itemize}
    \item но разные библиотеки и ОС позволяют себе всякие вольности;
    \item например, glibc разрешает longjmp из обработчика сигнала - можно
    использовать, чтобы реализовать потоки на уровне приложения, а не ОС.
  \end{itemize}
\end{itemize}
\end{frame}

\begin{frame}
\frametitle{Блокировка сигналов}
\begin{itemize}
  \item Некоторые сигналы можно блокировать
  \begin{itemize}
    \item сигнал не будет доставлен процессу пока его не разблокируют.
  \end{itemize}
  \item Для блокировки сигналов рекомендуется использовать sigprocmask
  \begin{itemize}
    \item функция принимает битовую маску и действие (например,
    заблокировать все сигналы в маске, или разблокировать).
  \end{itemize}
\end{itemize}
\end{frame}

\begin{frame}[fragile]
\frametitle{Пример}
\lstinputlisting[firstline=8,lastline=16,language=C]
{src/sigprocmask/sigprocmask.c}
\lstinputlisting[firstline=20,lastline=23,language=C]
{src/sigprocmask/sigprocmask.c}
\end{frame}
