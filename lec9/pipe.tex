\begin{frame}
\frametitle{Каналы (Pipes)}
\begin{itemize}
  \item Сигналы и коды возврата не позволяют передавать произвольные данные
  \begin{itemize}
    \item память у каждого процесса своя по-умолчанию;
    \item не пригодны для передачи данных - нет возможности обмениваться
    данными.
  \end{itemize}
  \item Для передачи данных между процессами UNIX IPC предоставляет много
  способов:
  \begin{itemize}
    \item можно создать участок общей памяти разделяемый процессами;
    \item сокеты - зачастую сеть, но есть специальные UNIX сокеты;
    \item файлы - можно читать/писать одни и те же файлы;
    \item каналы.
  \end{itemize}
\end{itemize}
\end{frame}

\begin{frame}
\frametitle{Каналы}
\begin{itemize}
  \item pipe - это пара файловых дескрипторов:
  \begin{itemize}
    \item один дескриптор можно использовать для записи;
    \item другой дескриптор можно использовать для чтения, того что было
    записано в первый;
    \item pipe имеет ограниченный буффер, так что записать в pipe не блокируясь
    произвольное количество данных не получится;
    \item pipe не сохраняет границы сообщений (по-умолчанию).
  \end{itemize}
\end{itemize}
\end{frame}

\begin{frame}[fragile]
\frametitle{Пример}
\lstinputlisting[firstline=9,lastline=14,language=C]{src/pipe/pipe.c}
\lstinputlisting[firstline=26,lastline=30,language=C]{src/pipe/pipe.c}
\lstinputlisting[firstline=32,lastline=33,language=C]{src/pipe/pipe.c}
\end{frame}
