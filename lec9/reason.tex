\begin{frame}
\frametitle{Назначение IPC}
\begin{itemize}
  \item Потоки vs процессы:
  \begin{itemize}
    \item потоки одного процесса разделяют все ресурсы - они зависимы;
    \item потоки разных процессов не разделяют ресурсов по-умолчанию.
  \end{itemize}
  \item Процессы по-умолчанию надежны:
  \begin{itemize}
    \item при правильном дизайне падение одного процесса не потревожит другой;
    \item упавший процесс всегда можно перезапустить.
  \end{itemize}
\end{itemize}
\end{frame}

\begin{frame}
\frametitle{Назначение IPC}
\begin{itemize}
  \item Процессам приходится взаимодействовать:
  \begin{itemize}
    \item чтобы выполнять общую задачу необходимо взаимодействовать;
    \item чтобы иметь доступ к общим ресурсам необходима синхронизация.
  \end{itemize}
  \item IPC - набор примитивов для взаимодействия процессов:
  \begin{itemize}
    \item обмен сообщениями;
    \item синхронизация.
  \end{itemize}
\end{itemize}
\end{frame}

\begin{frame}
\frametitle{Виды IPC}
\begin{itemize}
  \item Виды IPC определяются ОС и ее архитектуры:
  \begin{itemize}
    \item классические UNIX IPC;
    \item невнятные Windows IPC;
    \item L4 IPC (в каком-то смысле тоже классические).
  \end{itemize}
  \item Мы будем рассматривать классические UNIX IPC.
\end{itemize}
\end{frame}
