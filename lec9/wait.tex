\begin{frame}
\frametitle{Завершение процессов}
\begin{itemize}
  \item Завершение процесса состоит из двух частей:
  \begin{itemize}
    \item процесс должен завершиться или совершить ошибку;
    \item другой процесс должен дождаться пока процесс завершиться.
  \end{itemize}
  \item Чтобы завершиться процесс может вызвать exit или \_exit:
  \begin{itemize}
    \item как аргумент принимается некоторый код возврата;
    \item return из main за сценой приводит к вызову exit.
  \end{itemize}
\end{itemize}
\end{frame}

\begin{frame}
\frametitle{wait}
\begin{itemize}
  \item Чтобы дождаться завершения процесс может воспользоваться wait или
  waitpid:
  \begin{itemize}
    \item до тех пор пока кто-то не вызовет wait/waitpid процесс находится в
    состоянии \emph{zombie} - не жив и не мертв;
    \item зачастую wait вызывает процесс-родитель;
    \item если родитель умер, то ребенка усыновит/удочерит/etc другой процесс
    системы.
  \end{itemize}
  \item waitpid на самом деле дожидается не завершения процесса, а изменения его
  состояния:
  \begin{itemize}
    \item процессмы могу не только умирать, но и "останавливаться" и
    "продолжать".
  \end{itemize}
\end{itemize}
\end{frame}

\begin{frame}[fragile]
\frametitle{Пример}
\lstinputlisting[firstline=18,lastline=30,language=C]{src/wait/wait.c}
\end{frame}
