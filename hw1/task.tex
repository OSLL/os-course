\section{Основное задание}

\begin{enumerate}
  \item Инициализировать контроллер последовательного порта и создать функцию
  записи строки в последовательный порт.
  \item Настроить IDT и контроллер прерываний (изначально все прерывания должны
  быть замаскированы), в качестве теста включите функцию, которая генерирует с
  использование инструкции int, некоторое фиксированное прерывание, а обработчик
  прерывания должен печатать на экран сообщение.
  \item Настроит PIT на переодическую генерацию прерываний, в обработчике
  прерывания выводите в последовательный порт некоторое сообщение.
\end{enumerate}

\subsection{Примечания к основному заданию}

\begin{itemize}
  \item Третья часть задания, очевидно, не может быть принята без выполненной
  второй части задания.
  \item При выборе сообщений, которые вы будете выводить держите свою чувство
  юмора при себе, в противном случае можно получить наказание.
\end{itemize}

\section{Дополнительные задания}

\begin{enumerate}
  \item Реализуйте функцию вывода backtrace-а, т. е. функция в месте вызова
  должна вывести информацию о том, как исполнение кода дошло до места вызова.
  При этом функция не должна требовать для своей работы модификаций кода, т. е.
  нельзя, например, вставлять какой-то код в функции, чтобы backtrace мог их
  вывести.
  \item Реализуйте функции семейства print (а именно, printf, vprtintf,
  snprintf, vsnprintf). Функции должны поддерживать следующие модификаторы типа:
  \begin{itemize}
    \item d,i - целые знаковые числа в десятичном формате;
    \item u - целые беззнаковые числа в десятичном формате;
    \item o и x - целые беззнаковые числа в 8-ми ричном и 16-ти ричном формате;
    \item с - char;
    \item s - строка char-ов заканчивающаяся 0.
  \end{itemize}
  Кроме того должны поддерживаться следущие модификаторы размера:
  \begin{itemize}
    \item hh - hhd и hhi значат signed char, а hhu, hho и hhd - unsigned char;
    \item h - hd и hi значат short, а hu, ho, hx - unsigned short;
    \item l - ld и li значат long, а lu, lo, lx - unsigned long;
    \item ll - lld и lli значать long long, а llu, llo, llx - unsigned long
    long.
  \end{itemize}
  При этом нельзя ограничивать размер сообщения для функций printf и vprintf, т.
  е. нельзя их просто реализовать через vsnprintf или snprintf.
\end{enumerate}
