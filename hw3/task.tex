\section{Основное задание}

В этом домашнем задании вам необходимо научиться создавать/завершать/планировать
потоки исполнения, а также реализовать простую синхронизацию потоков исполнения.

С появлением потоков, вам придется внести изменения в уже написанный код, так
чтобы он стал потокобезопасным. Как миниум вам понадобится исправить алокатор
страниц и алокатор объектов фиксированного размера (Buddy и SLAB).

\begin{enumerate}
  \item Реализовать примитив взаимного исключения потоков (это очень простое
        задание);
  \item реализовать функции управления потоками, конкретный интерфейс остается
        на ваше усмотрение, главное чтобы можно было создавать новые потоки,
        завершать поток и дожидаться завершения других потоков;
  \item реализовать планировщик потоков и организовать \emph{вытесняющую}
        многозадачность; другими словами, реализовать функцию, которая
        приостанавливает текущий поток исполнения и вместо него ставит на
        процессор другой поток исполнения (если таковой имеется), и вызвать эту
        функцию из обработчика прерывания таймера, после того как поток
        выработает свой квант времени.
\end{enumerate}

В качестве примера интерфейса, который вам стоит реализовать вы можете
обратиться к POSIX threads library, или стандартному интерфейсу потоков в языках
C или C++.

\section{Дополнительные задания}

\begin{enumerate}
  \item Реализация блокирующего примитива взаимного исключения (то, что обычно
        называют mutex-ом). В отличие от примитива взаимного исключения в
        основном задании, если mutex был захвачен, когда поток попытался
        выполнить lock на нем, этот поток должен быть погружен в "сон" до тех
        пор, пока держатель mutex-а не выполнит на нем unlock\footnote{Имена
        функций и названия структур остаются на ваше усмотрение, но семантика
        погружения потока в сон должна оставаться.}.
  \item Реализовать условную переменную (conditional variable, для примера
        смотрите pthread\_cond\_t) и соответствующие функции для работы с ней:
    \begin{itemize}
      \item wait - ожидать, пока кто-нибудь не посигналит на условной
            переменной;
      \item signal - посигналить одному из потоков ожидающих на условной
            переменной;
      \item broadcast - посигналить всем потокам ожидающим на условной
            переменной;
    \end{itemize}
\end{enumerate}
