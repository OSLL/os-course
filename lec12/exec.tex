\begin{frame}
\frametitle{Исполняемые файлы}
\begin{itemize}
  \item Ядро ОС само по себе не очень полезно - пользователи имеют дело с
  прикладными приложениями
  \begin{itemize}
    \item соответственно, для полноценной ОС желательно уметь загружать и
    запускать программы.
  \end{itemize}
  \item Программы представляются как один или несколько исполняемых файлов:
  \begin{itemize}
    \item скрипты с аттрибутом +x и \#\! в начала файла;
    \item ELF файлы (опять же с аттрибутом +x);
    \item PE/COM файлы (это в Windows);
    \item Mach-O (это в Mac OS).
  \end{itemize}
\end{itemize}
\end{frame}

\begin{frame}
\frametitle{Бинарные исполняемые файлы}
\begin{itemize}
  \item Бинарный исполняемый файл содержит:
  \begin{itemize}
    \item код и данные необходимые для выполнения;
    \item ссылки на другие бинарные файлы (разделяемые библиотеки);
    \item может содержать отладочную информацию.
  \end{itemize}
  \item Бинарный исполняемый файл определяет
  \begin{itemize}
    \item где в памяти должны располагаться данные и код;
    \item где находится точка входа в программу (условно, адрес main).
  \end{itemize}
\end{itemize}
\end{frame}

\begin{frame}[fragile]
\frametitle{Формат ELF}
\begin{columns}
  \begin{column}{.36\linewidth}
    \begin{lstlisting}
struct elf_hdr {
    uint8_t e_ident[ELF_NIDENT];
    uint16_t e_type;
    uint16_t e_machine;
    uint32_t e_version;
    uint64_t e_entry;
    uint64_t e_phoff;
    uint64_t e_shoff;
    uint32_t e_flags;
    uint16_t e_ehsize;
    uint16_t e_phentsize;
    uint16_t e_phnum;
    uint16_t e_shentsize;
    uint16_t e_shnum;
    uint16_t e_shstrndx;
} __attribute__((packed));
    \end{lstlisting}
  \end{column}
  \begin{column}{.65\linewidth}
    \begin{itemize}
      \item ELF файл начинается с заголовка с общей информацией;
      \item тип, версия, архитектура - ОС должна проверить файл на валидность;
      \item адрес точки входа так же хранится здесь - \emph{e\_entry}.
    \end{itemize}
  \end{column}
\end{columns}
\end{frame}

\begin{frame}
\frametitle{Program Header}
\begin{itemize}
  \item Program Header, упрощенно, описывает участок памяти который должна
  подготовить ОС
  \begin{itemize}
    \item все Program Header-ы хранятся в таблице в ELF файле;
    \item при загрузке ELF файла, ОС должна прочитать эту таблицу и подготовить
    участки памяти.
  \end{itemize}
  \item Таблица Program Header-ов:
  \begin{itemize}
    \item смещение таблицы в файле хранится в поле \emph{e\_phoff};
    \item количество записей в таблице хранится в поле \emph{e\_phnum};
    \item размер каждой записи хранится в поле \emph{e\_phentsize}.
  \end{itemize}
\end{itemize}
\end{frame}

\begin{frame}[fragile]
\frametitle{Program Header для x86}
\begin{columns}
  \begin{column}{.65\linewidth}
    \begin{itemize}
      \item \emph{p\_type} - тип заголовка (на интересует только
      \emph{PT\_LOAD} == 1);
      \item \emph{p\_vaddr} и \emph{p\_memsz} - адрес и размер в памяти;
      \item \emph{p\_offset} и \emph{p\_filesz} - смещение и размер в файле
      данных, которые нужно загрузить в память;
      \item \emph{p\_filesz} может быть меньше \emph{p\_memsz}, хвост должен
      быть заполнен нулями.
    \end{itemize}
  \end{column}
  \begin{column}{.36\linewidth}
    \begin{lstlisting}
struct elf_phdr {
    uint32_t p_type;
    uint32_t p_flags;
    uint64_t p_offset;
    uint64_t p_vaddr;
    uint64_t p_paddr;
    uint64_t p_filesz;
    uint64_t p_memsz;
    uint64_t p_align;
} __attribute__((packed));
    \end{lstlisting}
  \end{column}
\end{columns}
\end{frame}

\begin{frame}
\frametitle{Загрузка ELF файла}
\begin{itemize}
  \item Таким образом загрузка ELF файла в простом случае состоит из следующих
  действий:
  \begin{itemize}
    \item проверяем заголовок ELF файла - что это ELF файл для нужной
    архитектуры;
    \item читаем таблицу Program Header-ов аллоцируем память и подготавливаем
    таблицу страниц для всех Program Header-ов с типом \emph{PT\_LOAD};
    \item передаем управление точке входа с переключением процессора в
    непривелигированный режим работы.
  \end{itemize}
\end{itemize}
\end{frame}
