\begin{frame}
\frametitle{Библиотеки}
\begin{itemize}
  \item Библиотеки позволяют переиспользовать код (зачастую без перекомпиляции):
  \begin{itemize}
    \item библиотеки можно организовывать разными способами, например:
    \begin{itemize}
      \item статические библиотеки - просто наборы объектных файлов, которые
      можно слинковать с вашей программой;
      \item динамические библиотеки подключаются во время работы программы.
    \end{itemize}
  \end{itemize}
  \item Преимущества динамических библиотек:
  \begin{itemize}
    \item не нужно хранить один и тот же код в нескольких экземплярах (в отличие
    от статических библиотек);
    \item можно избежать дублирования одного и того же кода в памяти (загрузить
    динамическую библиотеку в единственном экземпляре на все программы).
  \end{itemize}
\end{itemize}
\end{frame}

\begin{frame}
\frametitle{Динамические библиотеки}
\begin{itemize}
  \item Динамическая библиотека может быть загружена по любому адресу:
  \begin{itemize}
    \item почему мы не можем зафиксировать адрес?
    \item исполняемый файл не может заранее знать адреса функций и данных из
    библиотеки;
    \item сама библиотека не может знать адреса своих функций и данных заранее.
  \end{itemize}
\end{itemize}
\end{frame}

\begin{frame}
\frametitle{Position Independent Code (PIC)}
\begin{itemize}
  \item В момент сборки мы можем не знать абсолютные адреса функций/переменных,
  но мы можем знать относительные:
  \begin{itemize}
    \item функция которую мы хотим вызвать находится на \emph{x} байт выше/ниже
    текущей инструкции (значения регистра RIP);
    \item аналогично для данных.
  \end{itemize}
  \item PIC решает проблему обращения к функциям и переменным внутри
  динамической библиотеки
  \begin{itemize}
    \item если библиотека зависит от другой библиотеки - у нас все еще проблемы;
    \item программам/библиотекам зависящим от нас это не помогает.
  \end{itemize}
\end{itemize}
\end{frame}

\begin{frame}
\frametitle{Динамический компоновщик}
\begin{itemize}
  \item Динамический компоновщик решает ту же задачу, что и обычнй, но в других
  условиях
  \begin{itemize}
    \item в ELF файле может быть специальный Program Header с типом
    \emph{PT\_INTERP == 3} - он указывает на путь к динамическому компоновщику;
    \item вместе с имполняемым файлом ОС загружает динамический компоновщик
    \emph{и передает ему управление}.
  \end{itemize}
  \item Динамический компоновщик загружает нужные динамические библиотеки
  \begin{itemize}
    \item информация о необходимых библиотеках хранится в месте, на которое
    указывает Program Header \emph{PT\_DYNAMIC};
    \item т. е. в момент сборки мы должны знать обо всех библиотеках, от
    которых мы зависим.
  \end{itemize}
\end{itemize}
\end{frame}

\begin{frame}
\frametitle{Редактирование связей}
\begin{itemize}
  \item Динамический компоновщик нашел и загрузил все библиотеки в память, что
  дальше?
  \begin{itemize}
    \item теперь ему известны адреса всех нужных функций и переменных;
    \item он может подредактировать память и проставить в нужных местах ссылки.
  \end{itemize}
  \item Нужное место - Global Offset Table (GOT):
  \begin{itemize}
    \item по сути, таблица адресов всех объектов, к которым нам нужно
    обращаться;
    \item динамический компоновщик зная адреса просто заполняет GOT.
  \end{itemize}
\end{itemize}
\end{frame}

\begin{frame}[fragile]
\frametitle{Пример}
\begin{columns}
  \begin{column}{.3\linewidth}
    \begin{lstlisting}
int bar;

void foo(void)
{ ++bar; }
    \end{lstlisting}
  \end{column}
  \begin{column}{.6\linewidth}
    \begin{lstlisting}[style=base]
foo:
  push   %rbp
  mov    %rsp,%rbp

  -->mov    0x20093d(%rip),%rax // 0x200fd8<--
  -->mov    (%rax),%eax<--
  lea    0x1(%rax),%edx
  mov    0x200931(%rip),%rax
  mov    %edx,(%rax)

  nop
  pop    %rbp
  retq 
    \end{lstlisting}
  \end{column}
\end{columns}
\end{frame}

\begin{frame}[fragile]
\frametitle{Пример}
\begin{itemize}
  \item Куда указывает \emph{0x200fd8}? Посмотрим с помощью \emph{readelf -S}:
  \begin{lstlisting}
[Nr] Name Type     Address Offset Size EntSize Flags Link Info Align
...
[19] .got PROGBITS 200fd0  fd0    30   8       WA    0    0    8
...
  \end{lstlisting}
  \begin{itemize}
    \item динамический компоновщик после загрузки libfoo.so запишет в нужное
    место .got адрес переменной bar;
    \item а адрес .got находится с помощью относительной адресации.
  \end{itemize}
\end{itemize}
\end{frame}

\begin{frame}[fragile]
\frametitle{Пример}
\begin{columns}
  \begin{column}{.5\linewidth}
    \begin{lstlisting}[style=base]
main:
  push   %rbp
  mov    %rsp,%rbp

  mov    0x200930(%rip),%eax
  mov    %eax,%esi
  mov    $0x4007d4,%edi
  mov    $0x0,%eax
  callq  4005d0 <printf@plt>

  -->callq  4005f0 <foo@plt><--

  mov    0x200914(%rip),%eax
  mov    %eax,%esi
  mov    $0x4007e6,%edi
  mov    $0x0,%eax
  callq  4005d0 <printf@plt>

  mov    $0x0,%eax
  pop    %rbp
  retq
    \end{lstlisting}
  \end{column}
  \begin{column}{.4\linewidth}
    \begin{lstlisting}
int main(void)
{
  printf("bar = %d\n", bar);
  foo();
  printf("bar = %d\n", bar);
  return 0;
}
    \end{lstlisting}
  \end{column}
\end{columns}
\end{frame}

\begin{frame}[fragile]
\frametitle{Пример}
\begin{itemize}
  \item Что за функция \emph{foo@plt}?
\begin{lstlisting}
foo@plt:
  jmpq   *0x200a32(%rip) // 0x601028
  pushq  $0x2
  jmpq   4005c0 <_init+0x20>
\end{lstlisting}
  \begin{itemize}
    \item адрес \emph{0x601028} просто ссылается на определенное место в got;
    \item т. е. foo@plt просто передает управление по адресу записанному в got;
    \item в конечном итоге, там должен оказаться адрес функции foo.
  \end{itemize}
\end{itemize}
\end{frame}
