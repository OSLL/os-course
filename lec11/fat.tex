\begin{frame}
\frametitle{Простейшая файловая система}
\begin{itemize}
  \item Основная структура данных - связный список:
  \begin{itemize}
    \item мы будем связывать в список блоки фиксированного размера;
    \item размер блока кратен размеру сектора диска.
  \end{itemize}
  \item Файлы
  \begin{itemize}
    \item файл идентифицируется первым блоком;
    \item все блоки с содержимым файла просто связаны в список.
  \end{itemize}
  \item Каталоги
  \begin{itemize}
    \item каталог - файл, который хранит записи фиксированного формата
    \begin{itemize}
      \item каждая запись хранит имя дочернего каталога/файла, тип и номер
      первого блока.
    \end{itemize}
  \end{itemize}
  \item Свободные блоки
  \begin{itemize}
    \item все свободные блоки просто связаны в список.
  \end{itemize}
\end{itemize}
\end{frame}

\begin{frame}
\frametitle{"Эффективный" связный список}
\begin{itemize}
  \item Связный список не очень эффективная структура данных для индексации
  \begin{itemize}
    \item но мы все же постараемся сделать с этим что-нибудь...
    \item ну хоть что-нибудь...
  \end{itemize}
  \item Что нужно учитывать?
  \begin{itemize}
    \item общение с диском осуществляется секторами (минимум 512 байт);
    \item общение с диском медленное - чем меньше обращений тем лучше.
  \end{itemize}
\end{itemize}
\end{frame}

\begin{frame}
\frametitle{"Эффективный" связный список}
\begin{itemize}
  \item Заведем таблицу:
  \begin{itemize}
    \item каждая запись в таблице соответствует блоку ФС (i-ая запись,
    соответствует i-ому блоку);
    \item каждая запись - это просто число, а именно номер следующего блока или
    маркер конца списка.
  \end{itemize}
  \item Почему не хранить ссылку на следующий блок прямо в блоке?
  \begin{itemize}
    \item таблица гораздо компактнее - за одно чтение мы получаем сразу
    несколько ссылок;
    \item если повезет, то это будут нужные нам ссылки;
    \item если повезет, то мы можем прочитать всю таблицу целиком и хранить ее
    в памяти.
  \end{itemize}
\end{itemize}
\end{frame}

\begin{frame}
\frametitle{File Allocation Table (FAT)}
\begin{itemize}
  \item ФС FAT12/16/32 - пожалуй одна из самых популярных ФС в мире:
  \begin{itemize}
    \item активно используется во всяких устройствах (MP3 плееры, фотоаппараты,
    USB флешки и т.д. и т.п.);
    \item мало функциональная;
    \item не очень надежная;
    \item зато очень простая.
  \end{itemize}
  \item FAT использует связные списки и похожую таблицу блоков
  \begin{itemize}
    \item эта таблица и называется File Allocation Table (FAT).
  \end{itemize}
\end{itemize}
\end{frame}
