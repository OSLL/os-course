\begin{frame}
\frametitle{Файловая система}
\begin{itemize}
  \item Блочные устройства позволяют хранить большие объемы информации долгое
  время
  \begin{itemize}
    \item чем больше информации тем больше хочется ее структурировать.
  \end{itemize}
  \item Multics впервые (*) ввел в использование иерархическую файловую систему:
  \begin{itemize}
    \item файлы, каталоги/директории/папки, жесткие и символьные ссылки;
    \item так как Multics пофейлился, популярной иерархиеские ФС сделал скорее
    Unix.
  \end{itemize}
\end{itemize}
\end{frame}

\begin{frame}
\frametitle{Операции над файлами}
\begin{itemize}
  \item открытие файла - возвращает некоторый дескриптор, который используется
  для дальнейших операций с файлами
  \begin{itemize}
    \item снаружи этот дескриптор обычно выглядит просто как целое число;
    \item внутри ОС это обычно некоторая структура, которая кеширует информацию
    о файле (размер, права доступа, дата модификации и тд и тп);
  \end{itemize}
  \item закрытие файла - освобождение всех ресурсов связанных с открытым
  файлом;
  \item чтение и запись по некоторому смещеную в файле;
  \item запись за пределы файла изменяет его размер.
\end{itemize}
\end{frame}

\begin{frame}
\frametitle{Операции над каталогами}
\begin{itemize}
  \item создание и удаление файлов/жестких ссылок на файлы;
  \item создание и удаление символьных ссылок на файлы и каталоги;
  \item создание и удаление других каталогов;
  \item перечисление файлов/каталогов внутри каталогов.
\end{itemize}
\end{frame}
