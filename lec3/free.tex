\begin{frame}
\frametitle{Информация о доступной памяти}
\begin{itemize}
  \item До сих пор мы знали границы памяти, из которой мы удовлетовряем запросы
  на алокацию
  \begin{itemize}
    \item откуда берется это знание?
  \end{itemize}
  \item Для ядра ОС:
  \begin{itemize}
    \item из спецификации оборудования;
    \item от загрузчика/BIOS/UEFI или любого другого ПО, которое запускает ОС;
    \item например, multiboot загрузчик может предоставить карту памяти;
  \end{itemize}
  \item для обычных приложений:
  \begin{itemize}
    \item через интерфейс ОС;
    \item не редко ОС позволяет уменьшить/расширить границы памяти;
    \item например, в Unix-like системах есть вызовы \emph{brk}/\emph{sbrk},
    а так же \emph{mmap}/\emph{munmap}.
  \end{itemize}
\end{itemize}
\end{frame}
