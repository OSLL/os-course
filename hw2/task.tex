\section{Основное задание}

\begin{enumerate}
  \item Получить и вывести карту физической памяти. Добавить в карту памяти
  регион занятый ядром ОС и пометить его как занятый.
  \item Реализовать аллокатор страниц фищической памяти. Для аллокации вы можете
  использовать Buddy Allocator или придумать свой алгоритм. В любом случае
  учтите, что все аллоцированные страницы физической памяти должны быть
  выровнены на границу 4KB; естественно нужно уметь освобождать аллоцированные
  страницы.
  \item Реализовать аллокатор блоков памяти фиксированного размера, т. е. SLAB
  или свой собственный алгоритм; аллокатор должен удовлетворять следующим
  требованиям:
  \begin{itemize}
    \item аллокатор можно создать для любого размер от 1B до 4KB не
    включительно;
    \item аллокатор должен возвращать готовый к использованию указатель, или
    признак неудачи, если аллоцировать память не удалось;
    \item указатель должен находиться в верхней части адресного пространства (т.
    е. 47 бит должен быть равен 1);
    \item память, опять же, нужно уметь освобождать.
  \end{itemize}
\end{enumerate}

\section{Дополнительные задания}

\begin{enumerate}
  \item Реализуйте аллокатор памяти общего назначения (malloc/free, название
  функций можете выбрать на свой вкус);
  \item Реализовать отображение нескольких, возможно непоследовательных
  физических страниц в последоватльный участок логической памяти, при этом нужно
  учитывать следующее:
  \begin{itemize}
    \item диапазон логических адресов должен выбираться автоматически, а не
    передаваться как параметр, т. е. вы должны найти свободный диапазон
    логических адресов;
    \item нельзя использовать логические адреса в нижней части адресного
    пространства, т. е. адреса с 0-ым 47 битом.
  \end{itemize}
\end{enumerate}
